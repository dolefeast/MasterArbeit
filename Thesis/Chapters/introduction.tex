\chapter{Introduction}

Vacuum polarization is a fundamental phenomenon of quantum field theory (QFT), one of the first theoretical predictions of quantum field theory \cite{Uehling1935}\cite{Heis1936}. It shows how the vacuum also behaves as a dynamical medium, as a dielectric with permitivity $\epsilon > 1$.

For high enough external electromagnetic fields, the vacuumm will polarize inducing a non zero current density. The dependence of the current density on the applied electromagnetic field introduces non-linear terms in the Maxwell equation, leading to effects such as photon-photon scattering. Other interesting consequences of vacuum polarization is the screening behaviour of the vacuum, as (similarly to the behaviour of a classical dielectric) the polarized vacuum will act as a source for an electromagnetic field opposite to the one imposed.
Similarly to how a dielectric breaks under strong enough external electric fields, the Schwinger mechanism predicts the generation of real electron-positron pairs. However, the particles do not have to be generated to observe the non-linear effects of the vacuum. 

On another note, recent interests have arisen in the effects of considering boundary conditions for quantum field theory in the context of topological insulators.

\textbf{More things to talk about}
\begin{enumerate}
	\item The lack of a clearly defined vacumm state both in quantum field theories on Minkowski spacetime coupled to external electromagnetic fields and in free quantum field theories in generic background spacetimes lead to a decent analogy between these two. Studying and understanding the semiclassical Klein-Gordon-Maxwell equation might be a crucial step in understanding the semiclassical Klein-Gordon-Einstein equation, or the influence of the matter fields in the gravitational fields. Understanding semiclassical gravity might be the first of many steps to understand quantum gravity.
	\item The calcualtion of the vacuum polarization involves some ill-defined quantities, namely the product of distributions and thus require some renormalization. \cite{Ambjorn1983} do these exact calculations, as one naïvely would and get the wrong results \cite{Wendersson2022}
	\item In this project, we redo the calculations done in \cite{Ambjorn1983}, with the correct point-split renormalization w.r.t. a Hadamard parametrix. 
	\item We study the semiclassical approach to quantum field theory. This approach treats the background as a completely 
\end{enumerate}

\newpage
\section*{Notation and conventions}

Throughout this thesis, we use natural units $\hbar = c = 1$. We work on flat 1+1 dimensional Minkowski spacetime, with the `mostly negative' metric (+, -). $x$ will usually denote an event in the spacetime, with $x^{0}$, $x^{1}$ its time and spatial components, respectively. The partial derivatives with respect to these coordinates will then be 
\begin{align}
\partial_\mu := 	\frac{\partial}{\partial x^{\mu}}.
\end{align}
We also use a different coordinate choice,  which will describe the spacetime with coordinates $t$, $z$ for the time and spatial components, respectively. Similary 
\begin{align}
	\partial_t:= 	\frac{\partial}{\partial t}, \hspace{0.2cm}  \hfill
\partial_z:= 	\frac{\partial}{\partial z}.
\end{align}

Unless noted otherwise, Einstein's summation convention is assumed, i.e. 
\begin{align}
	a_\mu b^{\mu}= \sum_{n=0}^{N} a_\mu b^{\mu}.
\end{align}

