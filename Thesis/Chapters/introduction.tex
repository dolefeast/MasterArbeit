%\chapter{Introduction}
%
%Vacuum polarization is a fundamental phenomenon of quantum field theory (QFT), one of the first theoretical predictions of quantum field theory \cite{Uehl1935}\cite{Heis1936}. It shows how the vacuum also behaves as a dynamical medium, as a dielectric with permitivity $\epsilon > 1$.
%
%For high enough external electromagnetic fields, the vacuumm will polarize inducing a non zero current density. The dependence of the current density on the applied electromagnetic field introduces non-linear terms in the Maxwell equation, leading to effects such as photon-photon scattering. Other interesting consequences of vacuum polarization is the screening behaviour of the vacuum, as (similarly to the behaviour of a classical dielectric) the polarized vacuum will act as a source for an electromagnetic field opposite to the one imposed.
%Similarly to how a dielectric breaks under strong enough external electric fields, the Schwinger mechanism predicts the generation of real electron-positron pairs. However, the particles do not have to be generated to observe the non-linear effects of the vacuum. 
%
%On another note, recent interests have arisen in the effects of considering boundary conditions for quantum field theory in the context of topological insulators.
%
%\textbf{More things to talk about}
%\begin{enumerate}
%	\item The lack of a clearly defined vacumm state both in quantum field theories on Minkowski spacetime coupled to external electromagnetic fields and in free quantum field theories in generic background spacetimes lead to a decent analogy between these two. Studying and understanding the semiclassical Klein-Gordon-Maxwell equation might be a crucial step in understanding the semiclassical Klein-Gordon-Einstein equation, or the influence of the matter fields in the gravitational fields. Understanding semiclassical gravity might be the first of many steps to understand quantum gravity.
%	\item The calcualtion of the vacuum polarization involves some ill-defined quantities, namely the product of distributions and thus require some renormalization. \cite{Ambjorn1983} do these exact calculations, as one naïvely would and get the wrong results \cite{Wendersson2022}
%	\item In this project, we redo the calculations done in \cite{Ambjorn1983}, with the correct point-split renormalization w.r.t. a Hadamard parametrix. 
%	\item We study the semiclassical approach to quantum field theory. This approach treats the background as a completely 
%\end{enumerate}
%
%These results, appearing in the study of QFTCS, will still be useful to our case. There is some parallelism to be drawn between electromagnetism and gravitation. In General Relativity, the Levi-Civita connection gives rise to the Christoffel symbols that make the covariant derivative $\nabla_\mu = \partial_\mu + \Gamma^\nu_{\mu\rho}$ covariant. In QFTCS, one is interested in solving the Einstein Field equations 
%\begin{align}
%	R_{\mu\nu} + \frac{1}{2}g_{\mu\nu}R= 8\pi\left<T_{\mu\nu} \right>_\omega,
%\end{align} 
%where the field considered interacts with the spacetime through the expectation value of its stress-energy tensor. The solutions of the Einstein Field Equations are related to the curvature tensor of the connection $R^{\lambda}_{\alpha \beta \delta} = \left[ \nabla_\alpha, \nabla_\beta \right] $.
%Similarly, in the case of electrodynamics, the connection over the $U(1)$ fiber bundle gives rise to the covariant derivative $D_\mu = \partial_\mu+ ie A_\mu $, and the dynamical equation of interest is Maxwell's equations, $\partial_\mu F^{\mu\nu}=j^{\nu}$, with $F^{\mu\nu} = [D_\mu, D_\nu]$ the curvature of the connection. We therefore argue that the results from QFTCS can be used for our case, mainly by substituting the metric covariant derivative $\nabla_\mu$ by the gauge covariant derivative $D_\mu$.
%
%\newpage

\chapter{Introduction}

Quantum field theory describes the vacuum as a state with zero occupation number in all its modes, and it is usually portrayed in the literature as a sea in which virtual particle-anti-particle pairs are constantly being created and annihilated. This picture leads to an understanding of the vacuum no longer as the absence of matter, as it is the classical assumption, but as a dynamic and fluctuating entity governed by quantum effects. cf. \cite{Peskin:1995ev, srednicki} for further discussions on the quantization of fields and their vacuum states.

One of the first theoretically studied consequences of the quantum nature of the vacuum is vacuum polarization, which arises due to the presence of external electromagnetic fields. \cite{Dirac1934} developed the expression for the charge density of the Dirac field and extended the definition to include cases where background electromagnetic fields are present. 
The polarization of the vacuum leads to a field-strength dependent charge density of the vacuum, thereby turning vacuum electrodynamics into a non-linear theory. The consequences of this effect followed shortly, as was studied by \cite{Uehl1935} and \cite{Heis1936}.

This non-linearity in Maxwell's equations causes several observable phenomena, such as corrections to the Coulomb potential \cite{Uehl1935, PhysRev.101.843},  light-by-light scattering \cite{Heis1936, ATLAS:2016pab}, 
the Lamb shift \cite{Lamb1947}, Delbrück scattering \cite{Jarl1973, ATLAS:2016pab}, and the running of the fine structure constant \cite{2017485}. These mechanisms are also relevant as observational tools, e.g. the light-by-light scattering process leading to the attenuation of high energy photons is used in measuring the Hubble constant \cite{Domí2019} (cf. \cite{Fran2021} for a review). Additionally, the presence of strong magnetic fields (e.g. near the surface of neutron stars) leads to vacuum birefringence \cite{10.1093/mnras/stw2798}, where light propagating through the vacuum experiences polarization dependent refraction. 

Furthermore, \cite{Schw51} predicts the unobserved Schwinger mechanism, in which sufficiently strong electric fields can cause a breakdown of the quantum vacuum, creating electron-positron pairs.
%In the Dirac sea picture, this can be understood as holes being allowed to quantum-tunnel through the $2mc^2$ band gap and materialez.  
This mechanism requires electric fields of the order $E_C \sim 10^{18}V/m$ (cf \cite{Dunne2009} for a review), which sets the scale at which the \textit{perturbative} approximation of 4-dimensional QED breaks down. 

The above cited calculations were performed by in the so-called semi-classical approximation, where the matter field $\phi$ is quantized, but the background electromagnetic field under which $\phi$ propagates remains classical. In this approximation, the quantum fluctuations of $\phi$ are expected to be small and it couples to the background electromagnetic field through its back reaction, i.e. the expectation value of its charge density. Thus, when studying the coupling of charged quantum Klein-Gordon fields with classical electromagnetic fields, one speaks of solving the semi-classical Klein-Gordon-Maxwell equation 
\begin{subequations}
\begin{align}
    (D_\mu D^\mu + m^2 ) \phi = 0, \\
    \partial_\mu F^{\mu \nu} = j^\nu_C + \langle j^\mu_Q \rangle,
    \label{eq:maxwell}
\end{align}
\end{subequations}
with $j^\nu_C, j^\nu_Q$ the charge currents of the classical external sources and the quantum fields, respectively.
This approximation is said to hold when the fluctuations of the quantum field do not fluctuate excessively. 
\cite{Anderson_2003} proposes quantitative tests for the validity of the semi-classical approximation.

However, as already noted by Dirac, the standard construction of quantized fields cf. \cite{Peskin:1995ev} yields a very limited  algebra of observables. 
For a $\phi$ field, this includes only linear combinations of products of fields at separate spacetime points e.g. $\phi(x) \phi(y)$.  The operators one is interested in (e.g $j^\mu_Q$), lie outside of this algebra, as they involve terms quadratic in the fields. These operators need some sense of normal ordering to be defined.

Normal ordering proceeds by defining Wick polynomials e.g. $\phi^*\phi(x)$ through the two-point correlation function of the field
\begin{align}
    \omega^{\phi ^*\phi}_0 (x, y) = \bra{0} \phi^*(x) \phi(y) \ket{0} 
\end{align}
in the following fashion
\begin{align}
    \phi^*\phi(x) = \lim_{y\to x} \phi^*(x) \phi(y) - \omega^{\phi^* \phi}_0 (x, y) \mathbf{1}.
\end{align}

In the mode expansion of the field, $\phi = \sum (a_n + a_n^\dagger)$ this corresponds to reordering of the $a_n^\dagger$ operators to the left and the $a_n$ to the right in $\phi^2(x)$.
Notice the vacuum expectation value of this operator is always zero 
\begin{align}
 \langle \phi^2(x) \rangle_0 = \bra{0}  \left( 
 \lim_{y\to x} \phi(x) \phi^*(y) - \omega^{\phi \phi^*} (x, y) \mathbf{1}
    \right) \ket{0} = 0.
\end{align}
However, this is no longer an expected behavior of the field in the presence of external charges,  as it ignores the already discussed existence of vacuum polarization. 

In defining local non-linear operators, \cite{Dirac1934} assumed that the short-distance behavior of the vacuum in the presence of external charges was the same as the vacuum in their absence, up to smooth coefficients. In modern terminology, this assumption about the behavior of the state (particularly its two-point function) is called the Hadamard condition. States obeying this condition are regarded as the only physically meaningful, which allow the evaluation of Wick polynomials. See \cite{Schl2015} for an overview of Hadamard states in the presence of external potentials.

Similar problems arise in the study of Quantum Field Theory in Curved Spacetime (QFTCS), such as the definition of non-linear observables. Other problems of similar nature, which are not discussed in this work, include the ambiguity in defining a vacuum state in generic backgrounds. Over the past decades, substantial progress has been made in describing interactions in QFTCS \cite{Brunetti1996, Brunetti2000, Sahlmann2000, Hollands_2002, Hollands_2015, Brunetti_2003}.

These developments define observables in the language of distributions via the Hadamard condition, which was elegantly reformulated in the context of the modern branch of micro-local analysis \cite{Radzikowski1996}. This formulation enables  a rigorous definition of the \textit{large momentum behavior} of a distribution in a coordinate-independent manner. Within this context, it has been possible to define normal ordering and perturbative interacting quantum field theories in essentially the same way as on Minkowski spacetime. Further discussions on the Hadamard condition can be found in \cite{Wald1977, KAY199149, Fewster_2013}.

 Vacuum polarization does not lose significance in the context of QFTCS. It has been shown to avoid particle horizons and singularities \cite{Fulling1973, Fischetti1979}, mitigate anisotropies \cite{STAROBINSKY198099, Hartle1979}, drive inflation \cite{Guth1981},  and even possibly lead to closed universes \cite{Anderson:1985vi}.

Hadamard states are relevant both in the study of QFT with background electromagnetic fields and in curved spacetimes, as the definition of renormalized local observables should depend only on geometrically constructed data, such as the Hadamard parametrix. In QFTCS, this geometry is defined by the spacetime metric,  while in QFT with background electromagnetic fields, it is determined by that of the $U(1)$  bundle over spacetime.
Additionally, in both theories one is interested in the covariance of the quantities, therefore terms such as the Christoffel symbols or the gauge potentials cannot appear.
Further readings on the relation between QED in external fields and QFTCS can be found in \cite{marecki2004quantumelectrodynamicsbackgroundexternal}.

\cite{Ambj1983} studied the behavior of the charged (complex) free Klein-Gordon field  in the presence of external charges and how the inclusion of back reaction into the calculations affected the solutions. This study initially considers the external field approximation (i.e. by neglecting the back reaction of the scalar field) and displays the appearance of a critical background electric field strength, at which the energy of certain field modes become complex, leading to runaway solutions.
Moreover, it shows that including the back reaction of the scalar field screens the external charges, stabilizing the modes and thus avoiding complex mode energies.

However, it was later shown by \cite{Schl2015, Wernersson2020}, that their calculation of the vacuum polarization is flawed,
as it failed to include the parallel transport with respect to the gauge covariant derivative in the point-splitting calculation of the vacuum polarization. \cite{Ambj1983} also reported an unexpected results: under Neumann boundary conditions, vacuum polarization exhibits anti-screening behavior. \cite{Wernersson2020} corrects both of these calculations and presents the correct way of calculating the vacuum polarization in the presence of external charges. 

In this MSc. Thesis, we use this correct prescription to calculate the vacuum polarization, and study the behavior of the quantum charged Klein-Gordon field immersed in a classical background electromagnetic field, and how backreaction affects the solutions.

%\paragraph{Divergences of the quantum fields}
%divergences of QFT appear even in the consideration of the simplest theories. In the quantization of the free real scalar field, divergences appear when one tries to calculate any non-linear polynomial on the field, due to the presence of $a a^\dagger$ terms in the mode expansion of the field. In the absence of background electromagnetic field, this is usually dealt with by defining these observables through normal ordering, i.e. $\mathcal{O}(\phi) \to :\mathcal{O}(\phi):$. In this prescription, evaluation of an observable $:\mathcal{O}(\phi):$ in the state $\omega$ is defined by
%$$\left<: \mathcal{O}(\phi):\right>_\Omega = \bra{\Omega}  \mathcal{O}(\phi) \ket{\Omega}  - \bra{0}  \mathcal{O}(\phi) \ket{0}.$$
%this prescription is in itself very powerful in the context of QFT in the absence of background fields. 
%
%however, one of the key properties of normal ordering is yielding $\left<: \mathcal{O}(\phi):\right>_0=0$ for the vacuum state. This is no longer a desired behavior in the presence of background electromagnetic fields, as this would predict the vacuum polarization to be always zero, and is existence is already discussed.
%\cite{Dirac1934} proceeded by assuming that the vacuum state $\ket{0}_A$ in the presence of an electromagnetic field $A$ has the same short-range behavior as the $\ket{0}$ state in the absence of background electromagnetic field, up to smooth coefficients. That the vacuum state of the field is of this form is called the Hadamard condition, and the class of Hadamard states (i.e. states which verify the Hadamard condition) are usually said to be the only physically relevant states. 
%
%a \note{(quasi free?)} state is said to be Hadamard if its two-point function $w(x, x')$ is of the form 
%\begin{align}
%   w^{\phi \phi^*}_\Omega (x, x') = \bra{\Omega} \phi(x) \phi^*(x') \ket{\Omega} =  H^{\phi \phi^*}(x, x') + R^{\phi \phi^*}_\Omega(x, x').
%\end{align}
%here $H^{\phi \phi^*}(x, x')$, the Hadamard parametrix, is a state-independent bi-solution to the Klein-Gordon equation, with a divergence structure as $x' \to x$ very similar to that of the two-point function of the vacuum state in the absence of background electromagnetic fields. $R^{\phi \phi^*}_\Omega(x, x')$ is a smooth function that contains the dependence on the state $\Omega$.
%
%The Hadamard parametrix is calculated by integrating the Klein-Gordon equation along the geodesic that joins $x$, $x'$ (straight lines in flat spacetime), and by parallelly transporting the parametrix with respect to the gauge covariant derivative induced by the connection over the $U(1)$ principal bundle defined over the spacetime.
%\cite{Ambj1983} calculates the vacuum polarization without the inclusion of this parallel transport, which leads to a wrong expression as was shown by \cite{Schl2015, Wernersson2020}.


\newpage
\section*{Notation and conventions}

Throughout this thesis, we use natural units $\hbar = c = 1$. We work on flat 1+1 dimensional Minkowski spacetime, with the `mostly negative' metric (+, -). $x$ will usually denote an event in the spacetime, with $x^{0}$, $x^{1}$ its time and spatial components, respectively. The partial derivatives with respect to these coordinates will then be 
\begin{align}
\partial_\mu := 	\frac{\partial}{\partial x^{\mu}}.
\end{align}
We also use a different coordinate choice,  which will describe the spacetime with coordinates $t$, $z$ for the time and spatial components, respectively. Similary 
\begin{align}
	\partial_t:= 	\frac{\partial}{\partial t}, \hspace{0.2cm}  \hfill
\partial_z:= 	\frac{\partial}{\partial z}.
\end{align}

Unless noted otherwise, Einstein's summation convention is assumed, i.e. 
\begin{align}
	a_\mu b^{\mu}= \sum a_\mu b^{\mu}.
\end{align}


