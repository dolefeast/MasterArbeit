\chapter{Bibliography notes - to be deleted}
\begin{enumerate}
\item \cite{Eule1935} On the light-light scattering. Also develops a lagrangian for the electromagnetic field with vacuum polarization
\item \cite{Fran2021} review on light-by-light scattering in cosmology
\item \cite{Domí2019} on the effect of light-by-light scattering in cosmological measurements of the Hubble constant. Relevant in the $\gamma$-ray attenuation observations.
\item \cite{Anderson:1985vi} The vacuum energy density of the quantum fields can be large enough to close the universe
\item \cite{Hartle1979} The use of the effective-action method to calculate quantum effects in the early universe is described. An application is made to the calculation of the effect of one-loop contributions of conformally invariant matter fields on the evolution of homogeneous, isotropic, spatially flat classical geometries containing classical radiation.
\item \cite{Fulling1973} The back-reaction of the Klein-Gordon field can lead to avoidance of the cosmological singularity, at least on the time scale of the Friedmann equation.
\item \cite{STAROBINSKY198099} The Einstein equations with quantum one-loop contributions of conformally covariant matter fields are shown to admit a class of nonsingular isotropic homogeneous solutions that correspond to a picture of the Universe being initially in the most symmetric (de Sitter) state
\item \cite{Sahlmann2000}  In the setting of vector-valued quantum fields obeying a linear wave-equation in a globally hyperbolic, stationary spacetime, it is shown that the two-point functions of passive quantum states (mixtures of ground- or KMS-states) fulfill the microlocal spectrum condition (which in the case of the canonically quantized scalar field is equivalent to saying that the two-pnt function is of Hadamard form)
\item \cite{PhysRev.101.843} Calculates corrections to the Uehling term
\item \cite{Schwinger54} examines the effect of time independent external electromagnetic field on a Dirac field. The transformation function is constructed in a representation that adapts to the presence of this external field.
The Green's function is altered. 

Introduction of a modified vacuum state, which takes into account the external field. Calculates the modified vacuum energy and a formula is provided in terms of energy eigenvalues of discrete modes and phase shifts of continuum modes. Energy is written as a sum over discrete modes 

In \cite{Mohr1998} it is cited as \textit{AS a siemple example we note the Schwinger mechanism of spontaneous electron-positron creation in a homogeneuos electric field once the critical field strength of about E= $10^{16}V/cm$ has been exceeded}

\item \cite{Mueller1988} Studies positron creation in crossed-beam collosions of high-energy fully stripped heavy ions.The positron spectrum is shifted towards higher energies because of the absence of electron screening.For bare nuclei, pair creation will be enhanced by more than an order of magnitude the to the fact that the inner-shell states, in particular the K shell are avialable as a final state for the electron. This publication is cited \cite{Mohr1998} in the context of speculaqtions of nonlinear extensions of the original QED lagrangian

\item \cite{PhysRevD.32.1302} Studies solutions to the semiclassical backreaction equations for conformally invariant free quantum fields and a conformally coupled massive scalar field in spa5tially flat homogeneous and isotropic spacetimes containing classical radiation.

\textbf{From the introduction:} It is well established that quantum fields play a significant role in most models of the early universe. Their possible dynamical effects include the damping of anisotropy, the removal of particle horizons and singularities, and inflation. 

The dynamical effects of quantum fields
\begin{itemize}
    \item damps the anisotropies
    \begin{enumerate}
        \item  Ya. B. Zel'dovich and A. A. Starobinski, Zh. Eksp. Teor. Fiz. 61, 2161 (1971) [Sov. Phys. JETP 34, 1159 (1971)].
\item B. L. Hu and L. Parker, Phys. Rev. D 17, 933 (1978).
\item J. B. Hartle and B. L. Hu, Phys. Rev. D 20, 1772 (1979); 21, 2756 (1980).
    \end{enumerate}
\item removes of particle horizons and singularities 
\begin{enumerate}
    \item L. Parker and S. A. Fulling, Phys. Rev. D 7, 2357 (1973).
    \item M. V. Fischetti, J. B. Hartle, and B. L. Hu, Phys. Rev. D 20, 1757 (1979) 
    \item A. A. Starobinski, Phys. Lett. 91B, 99 (1.980).
    \item P. Anderson, Phys. Rev. D 28, 271 (1983).
    \item P. Anderson, Phys. Rev. D 29, 615 (1984 
\end{enumerate}
\item causes inflation
\begin{enumerate}
    \item A. H. Guth, Phys. Rev. D 23, 347 (1981)
    \item A. D. Linde, Phys. Lett. 108B, 389 (1982)
    \item A. Albrecht and P. J. Steinhardt, Phys. Rev. Lett. 48, 1220 (1982)
\end{enumerate}
\end{itemize}

\item \cite{hack2015cosmologicalapplicationsalgebraicquantum} Introduces the historical developments of AQFT, Hadamard states and studies the Standard Cosmological model and a fundamental study of perturbations in in Inflation.

\item \cite{Brunetti1996, Brunetti2000} define Wick polynomials for free and perturbative interacting theories in the context of microlocal analysis.

\item \cite{Hollands_2002} Establishes existence of local, covariant time ordered products of a scalar field in curved spacetime.

\item \cite{abramowitz+stegun} On the behavior of the modified Bessel function $K_0$ for small arguments
\begin{align}
    K_\alpha(x) = \int_0^\inf e^{-x cosh t} \cosh \alpha t dt
\end{align}

\end{enumerate}
