\chapter{Preliminaries}

\section{The gauge covariant derivative}


The Lagrangian of the complex Klein-Gordon field is usually introduced in QFT texts (\cite{Peskin:1995ev, srednicki}) as 
\begin{align}
    L = \partial_\mu \phi^* \partial^\mu \phi - m\phi^* \phi.
\end{align}
One easily checks that this lagrangian is invariant under global $U(1) = \{ e^{i\alpha} : \alpha \in \mathbb{R}\}$ gauge transformations
\begin{align}
    \phi(x) \to e^{i\alpha} \phi(x), \hspace{1cm}
    \phi^*(x) \to e^{-i\alpha} \phi^*(x),
\end{align}
but fails to be invariant under local $U(1)$ group actions, i.e. with non-constant $\alpha = \alpha(x)$, due to the kinetic term $\partial_\mu \phi^* \partial^\mu\phi$. Under local gauge transformations, this term picks up explicitly gauge dependent terms
$$-\partial_\mu \phi^* ' \partial^\mu \phi ' = 
\partial_\mu \alpha \partial^\mu \alpha\phi^* \phi
-i \partial_\mu \alpha \phi^* \partial^\mu \phi
+i \partial_\mu \alpha \partial^\mu\phi^*  \phi
+ \partial_\mu \phi^*\partial^\mu \phi
.$$

In fact, taking the derivative of the field along the direction of any unit vector $n$ involves comparing the field at different points,
\begin{align}
   \frac{\partial \phi}{\partial n}  = n^\mu \partial_\mu \phi = \lim_{\varepsilon \to 0} \frac{\phi(x+n\varepsilon) - \phi(x)}{\varepsilon}.
\end{align}
This expression does not present geometrical interpretation as the fields at $\phi(x+n\varepsilon)$ and at $\phi(x)$ have different transformation laws.
In order to compare the field at different points, the field at $x$, $\phi(x)$ should be parallely transported  to $y=x + \varepsilon n$ by the unitary transformation $U(y, x)$. The parallel transport condition equates to $U(y, x)$ having the following behavior under local gauge transformations
\begin{align}
U(y, x) \to e^{i\alpha(y)} U(y, x) e^{-i\alpha(x)},
\end{align}
together with the zero-distance condition $U(y, y)=1$.

We can check that $U(y, x) \phi(x)$ transforms correctly under gauge transformations 
\begin{align}
   U(y, x)\phi(x) \to e^{i\alpha(y)} U(y, x) e^{-i\alpha(x)} e^{i\alpha(x)} \phi(x) = e^{i\alpha(y)} U(y, x) \phi(x).
   \label{eq:gauge-transform-property}
\end{align}
$U(y, x)$ can be written as a pure phase $U(y, x) = e^{if(y, x)}$, and at first order in $\varepsilon$, where $\varepsilon$ is defined such that $\varepsilon n^\mu = (y-x)^\mu$, 
$$f(y, x) =  \varepsilon n^\mu \partial^{(y)}_\mu f(y, x)\lvert_{y=x} + O(\varepsilon)^2,$$
and therefore
\begin{align}
    U(y, x) = e^{if(y, x)} = 1 + i \varepsilon n^\mu \partial^{(y)}_\mu f(y, x)\lvert_{y=x} + O(\varepsilon^2).
\end{align}
Here, $\partial^{(y)}$ acts only on the $y$ variable.
Now with a proper comparison between the field at different points, the covariant derivative is defined along the direction $n$ by
\begin{align}
    \begin{split}
    n^\mu D_\mu \phi &= \lim_{\varepsilon \to 0} \frac{\phi(x+n\varepsilon) - U(x + n \varepsilon, x)\phi(x)}{\varepsilon} \\
    &= n^\mu\partial_\mu \phi + ie  n^\mu A_\mu \phi.
    \end{split}
\end{align}
This defines the gauge potential $A_\mu := - \frac{1}{e} \partial^{(y)}_\mu f(y, x)\lvert_{y=x},$ with the factor $e$ arbitrarily introduced.
Thus,
\begin{align}
    D_\mu = \partial_\mu + i e A_\mu.
\end{align}
The transform action \eqref{eq:gauge-transform-property} of $U(y, x)$  defines the gauge transformed $f(y, x)'$
\begin{align}
    U(y, x)'=e^{i\alpha(y)} e^{i f(y, x)} e^{-i\alpha(x)} =  e^{i( \alpha(y) + f(y, x) - \alpha(x))} = e^{if(y, x)'},
\end{align}
leading to the transformed gauge potential
\begin{align}
    A_\mu' = -\frac{1}{e}\partial_\mu^{(y)} f(y, x)' =- 
      \frac{1}{e}\partial_\mu^{(y)} (f(y, x) + \alpha(y))   = A_\mu - \frac{1}{e}\partial_\mu\alpha.
\end{align}
It is easy to check that $D_\mu$ is indeed covariant under $U(1)$ gauge transformations
\begin{align}
\begin{split}
    D_\mu \phi \to D_\mu'\phi' &= (D_\mu - i\partial_\mu \alpha) e^{i\alpha(x)}\phi \\
   &= (\partial_\mu + ie A_\mu - i\partial_\mu \alpha) e^{i\alpha(x)}\phi \\ 
   &= e^{i\alpha(x)}(i\partial_\mu\alpha(x) + \partial_\mu + ie A_\mu - i\partial_\mu \alpha) \phi \\ 
   &= e^{i\alpha(x)} D_\mu \phi(x).
\end{split}\end{align}
Equivalently, 
\begin{align}
    D_\mu' = e^{i\alpha(x)} D_\mu e^{-i\alpha(x)}.
    \label{eq:covariance}
\end{align}

This allows us to construct the $U(1)$ gauge invariant Klein-Gordon lagrangian 
\begin{align}
    L_{KG} = (D_\mu \phi)^* D^\mu \phi - m^2 \phi^* \phi.
\end{align}

Finally, in order to describe the kinematics of the gauge field $A_\mu$, we look at the covariance property \eqref{eq:covariance} of $D_\mu$ and notice that (gauge covariant) derivatives of the fields also transform as the field. In particular, one can indeed look at the commutator
\begin{align}
\begin{split}
   [D_\mu, D_\nu ] \phi &= [
   \partial_\mu + i e A_\mu, 
   \partial_\nu + i e A_\nu]\phi
   = ie([\partial_\mu , A_\nu] + [A_\mu, \partial_\nu]) \phi \\
   &= ie(
   \partial_\mu A_\nu 
    -  A_\nu \partial_\mu
   +A_\mu \partial_\nu
   -\partial_\nu A_\mu ) \phi \\
   &= ie((\partial_\mu A_\nu) + A_\nu\partial_\mu - A_\nu \partial_\mu + A_\mu\partial_\nu - (\partial_\nu A_\mu) - A_\mu \partial_\nu)\phi \\
   &=  ie(\partial_\mu A_\nu - \partial_\nu A_\mu ) \phi = ie F_{\mu\nu} \phi.
\end{split}\end{align}
Here we used $[\partial_\mu, \partial_\nu]\phi=[A_\mu, A_\nu]\phi=0$.

We identify $F_{\mu\nu} = \frac{1}{ie}[D_\mu, D_\nu]$ with the Faraday tensor, and we remark its gauge invariance
\begin{align}
\begin{split}
   F'_{\mu\nu}\phi' 
   = \frac{1}{ie}[D'_\mu, D'_\nu]  
   = \frac{1}{ie}
    [e^{i\alpha(x)} D_\mu e^{-i\alpha(x)}, 
    e^{i\alpha(x)} D_\nu e^{-i\alpha(x)}] e^{i\alpha(x)}\phi(x) \\
   =e^{i\alpha(x)}F_{\mu\nu}\phi 
   =F_{\mu\nu}\phi'.
\end{split}\end{align}
In the context of general gauge symmetries, $F_{\mu\nu}$ is the field strength tensor and is defined in the same fashion.

The simplest lagrangian describing the interaction between the Klein-Gordon field $\phi$ and the gauge field $A_\mu$ is the Klein-Gordon-Maxwell lagrangian,
\begin{align}
    L_{KGM} =  (D_\mu \phi)^* D^\mu \phi - m^2 \phi^* \phi - \frac{1}{4}F_{\mu\nu}F^{\mu\nu}.
\end{align}
This leads via the Euler-Lagrange equations to the Klein-Gordon-Maxwell equations, i.e. the coupling between the Klein-Gordon field and the electromagnetic field
\begin{subequations}
\begin{align}
    (D_\mu D^\mu + m^2)\phi &= (\eta^{\mu\nu}(\partial_\mu + ie A_\mu)(\partial_\nu + i e A_\nu) + m^2) \phi = 0 
    \label{eq:KG-equation}\\
    (D^*_\mu D^*^\mu + m^2)\phi^* &= (\eta^{\mu\nu}(\partial_\mu - ie A_\mu)(\partial_\nu - i e A_\nu) + m^2) \phi^* = 0 \\
    \partial_\mu F^{\mu \nu} &= ie (\phi^* D^\nu \phi - \phi D^\nu^* \phi^*) = j^\nu.
    \label{eq:maxwell-equation}
\end{align}
\end{subequations}
Here, we directly see how the Klein-Gordon field acts as a source for the electromagnetic field through its Noether current $j^\nu$ defined in Equation \eqref{eq:maxwell-equation}.

This construction of gauge invariant lagrangians can be more elegantly stated and generalized to Lie group actions to construct the Yang-Mills lagrangian, cf \cite{Hamilton:2017gbn}.
\section{Quantization of the free complex scalar field}
\label{sec:quantization}

We study the Klein-Gordon field subject to the equation of motion \eqref{eq:KG-equation}.
When submitted to Dirichlet 
\begin{align}
    \phi(t, 0) = \phi(t, a) = 0,
\end{align}
or  Neumann 
\begin{align}
   \left. \partial_1 \phi(t, x)  \right|_{x=0}
   =\left. \partial_1 \phi(t, x)  \right|_{x=a} = 0
\end{align}
boundary conditions,
the general solution to equation \eqref{eq:KG-equation} can be expressed as a sum of energy modes
\begin{align}
	\phi(t, x) = \sum_{n>0}^{} a_n \phi_n (x) e^{-i\Omega_n t}
+\sum_{n<0}^{} b^*_n \phi_n (x) e^{-i\Omega_n t},
\label{eq:field-expansion}
\end{align}
with $a_n, b_n$ complex coefficients depending on $n $, and $\phi_n$ solutions to the time independent Klein-Gordon equation
\begin{align}
	\left[ \left( \Omega_n - eA_0 \right)^2 + \partial_1^2 - m^2  \right] \phi_n(x) = 0.
    \label{eq:TIKGE1}
\end{align}
Here, $\Omega_n$ are gauge dependent energy parameters that are calculated when applying the boundary conditions $\phi$ is subjected to. The $\phi_n$ are orthonormal with respect to the inner product 
\begin{align}
	(\phi_n, \phi_m) = i\int_{0}^{1} \left(  \phi_n(x)^*D_0\phi_m(x) -  \phi_m(x)D_0^*\phi_n(x)^* \right) dx,
\end{align}
and $n>0$ corresponds to modes of positive norm and $n<0$ corresponds to modes of negative norm.

The canonical momentum $\pi$ associated to the field $\phi$ is calculated as
\begin{align}
	\pi(x) = D_0^*\phi^*= i\sum_{n>0}^{} a^*_n(\Omega_n - eA_0)\phi_n^*(x) e^{i\Omega_n t}
+ i\sum_{n<0}^{} b_n(\Omega_n - eA_0)\phi_n^*(x) e^{i\Omega_n t},
\end{align}
and equivalently for the momentum $\pi^*$ associated to $\phi^*$ is  
\begin{align}
	\pi^*(x) = D_0\phi= -i\sum_{n>0}^{} a_n(\Omega_n - eA_0)\phi_n(x) e^{-i\Omega_n t}
- i\sum_{n<0}^{} b_n^*(\Omega_n - eA_0)\phi_n(x) e^{-i\Omega_n t}.
\end{align}

Canonical quantization proceeds by promoting the classical fields $\phi, \pi$ to operators on the state Hilbert space obeying the same-time commutation relations 
\begin{align}
	[\phi(t, x), \pi(t, y)] = 
	[\phi^*(t, x), \pi^*(t, y)] = i \delta(x-y)\mathbf{1}.
\end{align}
The imposition of these commutation relations, together with the orthonormality and completeness condition, imply that the operators corresponding to the classical coefficients $a_n$, $b_n$ are the ladder operators obeying the commutation relations
\begin{align}
	[a_n, a_m^\dagger] &= [b_n, b_m^\dagger]= \delta_{nm}\mathbf{1}, \\
	[a_n, a_m] = [b_n, b_m]&= 
	[a_n, b_m] = [a_n, b^\dagger_m]=  0.
\end{align}
The ladder operators, in particular the annihilation operators $a_n, b_n $ define the vacuum state as that element of the Hilbert space of states which obeys 
\begin{align}
	a_n \ket{0} = b_n \ket{0} = 0.
\end{align}

One also talks about the two two-point functions for a particular state $\chi$\footnote{The two point functions are also defined distributionally by 
\begin{align*}
	w^{\phi \phi^*}(f, g) = \bra{\Omega} \phi(f)\phi^*(g) \ket{\Omega} = \int_{\mathbb{R}^2\times \mathbb{R}^2}^{}  w^{\phi\phi^*}_\Omega(x, x') f(x) g(x')dx dx'
\end{align*}}
\begin{align}
	w_\chi^{\phi \phi^*}(t, x; t', x') &= \bra{\chi} \phi(t, x) \phi^*(t', x') \ket{\chi}  \\
	w_\chi^{\phi^* \phi}(t, x; t', x') &= \bra{\chi} \phi^*(t, x) \phi(t', x') \ket{\chi}.
\end{align}
The two-point function is a bi-solution of the Klein-Gordon equation
\begin{align}
	\mathcal{K}w_\chi^{\phi \phi^*}(t, x; t', x')  
	= \mathcal{K'} w_\chi^{\phi \phi^*}(t, x; t', x')   = 0.
\end{align}
In the previous equation $\mathcal{K} = D_\mu D^{\mu} + m^2$ the Klein-Gordon operator, and $\mathcal{K'}$ the Klein-Gordon operator acting on the variables $(t', x')$.

In particular for the vacuum state, the two-point functions present the following form
\begin{align}
	w_0^{\phi \phi^*}(t, x; t', x') &= \sum_{n>0}^{}  \phi_n(x)\phi_n^*(x') e^{-i\Omega_n(t-t')} \\
		w_0^{\phi^*\phi }(t, x; t', x') &= \sum_{n<0}^{}  \phi_n(x)\phi_n^*(x') e^{i\Omega_n(t-t')}.
        \label{eq:two-point-function}
\end{align}


Finally, measurement of a linear observable $\mathcal{O}(\phi)$ is done by taking its expectation value with respect to some state $\Omega$
\begin{align}
	\left< \mathcal{O}(\phi) \right>_\Omega =\bra{\Omega} \mathcal{O}(\phi) \ket{\Omega} .
\end{align}
Observables $\mathcal{O}(\phi)$ non-linear on the fields involve taking the product of the distributions $\phi$. This operation is a priori ill-defined and its proper mathematical definition is dealt with in the next section.

%\subsection{Short-distance behavior of $w$}
%
%In this work we will be particularly interested in studying the short distance behavior of the two-point function of the massless Klein-Gordon field in the absence of electromagnetic fields.
%For Dirichlet boundary conditions, the mode solutions to Equation \eqref{eq:TIKGE1} are 
%\begin{align}
%    \omega_n = \frac{n\pi}{a}, \hspace{1cm} \phi_n(x) = \sqrt{\frac{1}{n \pi}}\sin \frac{n \pi x}{a}.
%\end{align}
%Substituting this into the two-point functions in the Equations \eqref{eq:two-point-function} yields 
%\begin{align}
%    	w_0^{\phi \phi^*}(t, x; t', x') &= \sum_{n>0}^{} \frac{1}{n\pi} \sin \frac{n \pi x}{a}\sin \frac{n \pi x'}{a}  e^{-in(t-t')}.
%\end{align}
%We focus on the splitting along the time direction, i.e. $x'=x, t'=t + \tau$, thus
%\begin{align}
%    	w_0^{\phi \phi^*}(\tau) := 
%    	w_0^{\phi \phi^*}(x, t; x, t + \tau) 
%        &= \sum_{n>0}^{} \frac{1}{n\pi} \sin^2 \frac{n \pi x}{a}  e^{in\pi\tau}.
%\end{align}
%If we look at the time derivative of $w(\tau)$, 
%\begin{align}
%    	\partial_\tau w_0^{\phi \phi^*}(\tau) := 2\sum_{n>0} e^{in\pi\tau} \sim  \frac{2}{1- e^{i\pi \tau}} = \frac{2}{i\pi \tau}.
%\end{align}
%Integrating,
%\begin{align}
%    	 w_0^{\phi \phi^*}(\tau) := \frac{2}{i\pi} \log \tau,
%\end{align}
%\note{which is nonsense but has to be looked into}
\section{Hadamard two point functions. Point-split renormalization.}
\label{sec:Hadamard}


The construction described in the section above presents an observable algebra that is limited to linear combinations of products of fields at different points, but fails at describing monomials e.g.~$\phi^\dagger \phi(x)$. In particular, we are interested in calculating the vacuum polarization
\begin{align}
	\rho(x) = \left<\hat{\rho}(x) \right>_\Omega= ie\bra{\Omega} \phi^* D_0\phi(x) - \phi D_0^* \phi^*(x) \ket{\Omega}.
	\label{eq:rho-expectation-value}
\end{align}
Due to the distributional nature of the field this expression is ill-defined.

It might be illustrative to consider the mode expansion for $\phi$ in Equation \eqref{eq:field-expansion}, where one sees that the presence of  the $a_n a^\dagger_n$, $b_n b_n^\dagger$ terms lead to divergences.  A powerful tool in removing these divergences is normal ordering, which defines field operators using their zero-point value\textemdash the vacuum expectation value\textemdash as a reference point. The charge density at some point $x$ for a the field in a state $\Omega$ is therefore calculated as
\begin{align}
	\left<:\hat{\rho}(x) :\right>_\Omega:= \bra{\Omega} \hat{\rho}(x) \ket{\Omega} - \bra{0} \hat{\rho}(x) \ket{0}.
\end{align}
One immediately sees that under this prescription, the vacuum polarization is always 0, and therefore normal ordering ceases to be an appropriate prescription in the presence of electromagnetic fields.  Normal ordering also fails to account for effects such as the Casimir force. 


A well-defined way of dealing with these point-wise products of fields is by point-split renormalization with respect to a Hadamard parametrix. Physically reasonably states  present a two-point function $w^{\phi \phi^*}(x, x')$ of Hadamard form \cite{Wroc2011}
\begin{align}
    w_\Omega^{\phi \phi^*}(x, x') = 
    H^{\phi \phi^*}(x, x') + R_\Omega^{\phi \phi^*}(x, x')
\end{align}
with $H^{\phi \phi^*}$ the Hadamard parametrix, a divergent bi-distribution independent of the state $\Omega$, resembling the behavior of the vacuum two-point function in absence of external fields, up to smooth coefficients, and $R_\Omega^{\phi\phi^*}$ a smooth function which does depend on the state.

The Hadamard parametrix is a state independent bi-distribution, which in general takes the form
\begin{align}
H^{\phi\phi^*}(x, x') = \sum_{k=0}^{} V_k(x, x') T_k(x, x')
\label{eq:hadamard-expansion}
\end{align}
with $V_k(x, x')$ smooth coefficients to be found. In 1+1 spacetime dimensions\footnote{cf.  \cite{D_canini_2008} for treatments of the Hadamard parametrix in $n\neq 2$ spacetime dimensions.}, the $T_k$ take the form
\begin{align}
	T_k(x,x') = 
	-\frac{1}{ 2^{2+2k}\pi k!} (x-x') ^{  -2k} \log \frac{-(x-x')_\epsilon^2}{\Lambda^2}  
\end{align}
with  
\begin{align}
	 (x-x')^2_\varepsilon = (x-x')^2 - i\varepsilon \text{sign} (x-x')^0
\end{align}
and $\Lambda>0$ an arbitrary scale. Changes in $\Lambda$ amount to smooth modifications in the parametrix, which get absorbed by the smooth function $R$.
It is important to note that the Hadamard parametrix is constructed entirely by geometrical means and is independent of the state. 


To find the $V_k( x, x')$  functions, one imposes that $w(x, x')$ is a bi-solution to the Klein-Gordon equation, i.e. 
\begin{align}
	(D_\mu D^\mu  + m^2)w^{\phi \phi^*}_\Omega(x, x') = ( D^*'_\mu D^*'^{\mu} + m^2)w(x, x') = 0,
\end{align}
with $D'^*_\mu = \partial'_\mu - ieA_\mu(x')$ acting on  $x'$.
With this imposition, one finds the Hadamard recursion relations (\cite{dewitt1960radiation, Schl2015})
\begin{align}
    (x-x')^\mu D_\mu V_k(x, x') + k V_k(x, x') = -k \mathcal{K} V_{k-1}(x, x')
    \label{eq:transport-equation}
\end{align}
which together with the initial conditions \newline $\lim_{x' \to x} V_0(x, x') = 1$ leads to the family $V_k$ of solutions. 

In practice, the full expansion of the parametrix is not needed (in fact, it does not converge in general\footnote{The Hadamard expansion should rather be read in a way that when truncated after some $k\geq \frac{n}{2}$, $W(x, x') \in C^{k+1 -\frac{n}{2}}$.}) since the $T_j$ vanish increasingly fast as $x'\to x$. The charge density operator of the Klein-Gordon field only contains first order derivatives of the field, and thus we are only interested in the $k=0$ term of the expansion, containing the $V_0$ coefficient, higher order terms vanish in the coinciding points limit.

Once $H^{\phi \phi^*}$ is known, one can define the expectation value of point-wise products of fields (and derivatives) through point-splitting with respect to the Hadamard parametrix, 
\begin{align}
	\left<D_\alpha \phi D^*_\beta \phi^*(x) \right>_\Omega = \lim_{x' \to x} D_\alpha D_\beta^* ' \left( w^{\phi\phi^*}_\Omega(x, x') - H^{\phi \phi^*}(x, x') \right),
	\label{eq:point-splitting-wrt-a-Hadamard-parametrix}
\end{align}
with $\alpha, \beta$ symmetrized multiindices.
Due to the Hadamard form of the two-point function, in the limit of coinciding points, the divergences cancel out, resulting in a well-defined expectation value.
%\paragraph{Schrödinger vs. Heisenberg picture}
%In non-relativistic quantum mechanics (this can also be done in QFT, but it is non-standard), the usual way of studying solutions to the Schrödinger equation is using the Schrödinger picture.
%The Schrödinger picture means that solutions to the Schrödinger equation 
%\begin{align}
%	i\hbar \partial_t \psi(t, \textbf{x}) = H \psi(t, \textbf{x})
%\end{align}
%evolve in time as follows 
%\begin{align}
%	\psi(t, \textbf{x}) = e^{-iH (t-t_0)} \psi(t_0, \textbf{x}),
%\end{align}
%i.e. it focuses in the study of the time evolution of the wave function, where the operators remain "constant" in time. In this picture, the evaluation expectation values of some observable in the Schrödinger picture $\mathcal{O}_S$ and its evolution in time goes as 
%\begin{align}
%	\bra{\phi(t, \textbf{x})} \mathcal{O}_S \ket{\phi(t, \textbf{x})}  = 
%	\bra{\phi(t_0, \textbf{x})} e^{iH(t-t_0)} \mathcal{O}_S e^{-iH(t-t_0)} \ket{\phi(t_0, \textbf{x})} .
%	\label{eq:time-evolution-in-the-schrodinger-picture}
%\end{align}
%One might equivalently work in the Heisenberg picture, in which the study focuses on the time evolution of the operators, and the vectors are time-independent. By observing \eqref{eq:time-evolution-in-the-schrodinger-picture}, the operator $\mathcal{O}$ in the Heisenberg picture is defined as 
%\begin{align}
%	\mathcal{O}_H = e^{iH(t-t_0)} O_S e^{-iH(t-t_0)},
%\end{align}
%and differentiation of this expression with respect to time yields the equivalent time evolution equation for the operator $\mathcal{O}$ in the Heisenberg equation 
%\begin{align}
%		\frac{d\mathcal{O}_H}{dt} = i[H, \mathcal{O}_H].
%\end{align}
%

%When solved with compactly supported Cauchy data $\left(\left. \phi\right|_\Sigma, \left. n^{\mu}D_\mu\phi \right|_\Sigma\right)$, with $\Sigma$ a space-like hypersurface (typically a constant-$t$ surface) and $n$ a unit vector orthogonal to $\Sigma$, the global hyperbolicity of Minkowski spacetime ensures that this problem is well posed.
%		The collection of all possible smooth $\phi$ with compactly supported Cauchy data forms the configuration space $\mathcal{C}$. From $\mathcal{C}$ one constructs the phase space $\mathcal{P} = T^*\mathcal{C}$, as its cotangent space. The elements of $\mathcal{P}$ are the points $(\phi, \pi)$, with $\pi=D_0^*\phi^*$ the canonical conjugate of the field $\phi$. Points in the phase space correspond to initial conditions for the Klein-Gordon equation on the Cauchy surface $\Sigma.$
%		$\mathcal{P}$ can be endowed with symplectic structure with the symplectic\footnote{Bilinear, antisymmetric} form
%		\begin{align}
%			\sigma((\phi_1, \pi_1), (\phi_2, \pi_2)) = i \int_\Sigma \left( \pi_1^* \phi_2^*- \pi_2\phi_1 \right) d\Sigma.
%		\end{align}
%		It is not difficult to prove that $\sigma$ is a time evolution invariant if $\phi_1$, $\phi_2$ are solutions to the Klein-Gordon equation, as long as they are compactly supprted. \note{Prove this? In appendix?}
%	
%The solution space  $\mathcal{S}$ is the space of all smooth solutions to the Klein-Gordon equation. As the initial data fully determines a solution to the Klein-Gordon equation, each solution corresponds to a point in phase space and therefore $\mathcal{S}$ can be identified with $\mathcal{P}$.
%$\mathcal{S}$ can be given Hilbert space structure as follows: 
%\begin{enumerate}
%	\item Define the subspace $\mathcal{S}_+$ of positive frequency solutions. In this subspace the symplectic product can be made positive definite.
%	\item Define a Hermitian inner product in this subspace 
%	\begin{align}
%		(\phi_1, \phi_2) = i\sigma(\phi_1^*, \phi_2), \,\,\, \phi_1, \phi_2 \in S_{+}
%	\end{align}
%\item Complete $\mathcal{S}_+$ with respect to this norm. \note{How to go from this to the whole $\mathcal{S}$?}
%\end{enumerate}
%	
%
%In classical mechanics, observables are the compactly supported functions \linebreak $f: \mathcal{S}\rightarrow \mathbb{C}$. They can be equipped with an algebra structure via the Poisson brackets, 
%\begin{align}
%\{f, g\}  := \int_{\Sigma}^{}  \left( \frac{\delta f}{\delta \phi} \frac{\delta g}{\delta \pi} - \frac{\delta g}{\delta \phi} \frac{\delta f}{\delta \pi}  \right) d\Sigma.
%\end{align}
%
%First quantisation is done by the\, $\hat{}$\, operator, which elevates classical observables $f$ to operators $\hat{f}$ on the $\mathcal{S}$ obeying the algebraic rule 
%\begin{align}
%	[ \hat{f}, \hat{g} ] = i \hat{\{f, g\} },
%\end{align}
%with $ \hat{\{f, g\} }$ the quantized version of the Poisson bracket.
%
%The field observables are defined with respect to smooth compactly supported test functions $f$
%\begin{align}
%	\phi(f) = a(f)e^{-i\omega t} + a^\dagger(\overline{f}) e^{i\omega t},
%\end{align}
%where the $a$ and $a^\dagger$ operators are defined the same way the are constructed in the quantum harmonic oscillator problem. 
%
%The measured observables of the field in a state $\Omega \in \mathcal{F}(\mathcal{S)}$ are therefore the expectation value $ \bra{\Omega} \phi(f) \ket{\Omega}$.
%Through $\phi(f)$ one can define the field as the weighted average by the test function $f$ of the spacetime distribution $\phi(x)$ through 
%\begin{align}
%	\phi (f) = \int_{\mathbb{R}^{n}}^{} \phi(x) f(x) d^{n}x,
%\end{align}
%with $\phi(x)$ an operator valued distribution.
%This also formally defines the two-point function as the distributional kernel of
%\begin{align}
%	w^{\phi\phi^*}_\Omega(f, h) 
%	= \bra{\Omega} \phi(f)\phi^*(g) \ket{\Omega}  
%	= \int_{\mathcal{M}\times \mathcal{M}}  f(x) g(x') w^{\phi\phi^*}_\Omega(x, x') dx dx'.
%	\label{eq:two-point-function-formal-definition}
%\end{align}



%\textbf{Requirements}: The GNS construction, the definition of state.
%
%Given an algebra of observables $\mathcal{A}$, an algebraic state $\omega$ is defined as the linear map $\omega: \mathcal{A} \to \mathbb{C}$ which obeys (certain properties: unitarity, positive definite, some more).  Given two elements $f, h$ of the phase space, the two point function of a state is defined as 
%\begin{align}
%	w(f, h) = \omega(\sigma_f \sigma_h).
%\end{align} 
%One can also define the kernel $w(x, x')$ of the two point function (which will also be called the two point function) through 
%\begin{align}
%w(f, h) = \int_{\mathcal{M} \times \mathcal{M}}^{} w(x, x') f(x) g(x') dx dx'
%\end{align}
%
%\newpage
%\section{The basic idea}
%
%\note{This is the basic idea. A lot of concepts need to be introduced before this.}
%
%\ldots therefore fields are to be understood as smeared operators over some test function, as
%\begin{align}
%	\phi(f) = \int_{\mathcal{M} }^{}  \phi(x) f(x) dx.
%\end{align} 
%
%\newpage
% 
%\section{Things that might be useful in this chapter}
%
%The quantities involved in the currents $T_{\mu \nu}$ or $j^\mu$ are all of the form 
%\begin{align}
%	\left(
%	D_\alpha \phi
%	\right) ^* D_\beta \phi,
%\end{align}
%with $\alpha, \beta$ two symmetrized multiindices. Terms with more derivatives are `more singular'.
%
%To calculate the expectation value of this operator, we begin by assuming as Dirac did \cite{Dirac1934}, that all physically relevant states are Hadamard. Namely that their two point functions with respect to a state $\Omega$
%\begin{align}
%	w^{\phi \phi^*}_\Omega(x, x') = \bra{\Omega}\phi(x) \phi^*(x')\ket{\Omega}
%\end{align}
%have the same singularity structure with $x' \to x$ as it would on a flat background.  More precisely, this two point function can be written in the form
%\begin{align}
%	w^{\phi \phi^*}_\Omega(x, x') = H^{\phi \phi^*}(x, x')+ R_\Omega^{\phi \phi^*}(x, x'),
%\end{align}
%with $H^{\phi \phi ^*}$ the Hadamard parametrix, and $R_\Omega ^{\phi\phi^*}$ a smooth function as $x' \to x$.  The Hadamard parametrix is singular as $x' \to x$, and it is state independent
%\footnote{Conversation with Rishabh (19/7/24): An incorrect renormalization scheme induces gauge dependent results, as what we are interested in is the $R_\Omega^{\phi\phi^*}$. Note that assuming Hadamard states, the gauge dependence of both $\omega_\Omega$ and $H$ cancels out when substracted.}.
%
%The assumption of the state being Hadamard allows us to define the expectation value of the quadratic expression of interest following a point-splitting scheme, i.e. splitting the point at which we want to evaluate the expectation value, and taking the limit of coinciding points,
%\begin{align}
%	\bra{\Omega} D_\alpha\phi(x)(D_\beta\phi)^*(x)\ket{\Omega} :=
%	\lim_{x'\to x }\left[ D_\alpha D'_\beta^*\left( 
%	w^{\phi \phi^*}_\Omega(x, x') - H^{\phi \phi^*}(x, x')
%	\right)  \right].
%	\label{eq:point-split}
%\end{align}
% $\alpha, \beta$ are symmetrized multiindices and $D'_\mu^* = \partial'_\mu - ieA_\mu$ the application of the covariant derivative on $x'$.
%Substracting the state independent singular part of the Hadamard parametrix from the two point function yields the evaluation of the regular function $R_\Omega^{\phi \phi^*}$.
%
%For the massless case \note{Still need to calculate the leading order expansion for the massive case}, the Hadamard parametrix is
%\begin{align}
%	H^{\phi \phi^*}(x, x') = U(x, x') \log \left( - (x - x')^2 + i\epsilon (x - x')^{0} \right) ,
%\end{align}
%with $U(x, x')$ a smooth function given by the parallel transport with respect to the gauge covariant derivative 
%\begin{align}
%	U(x, x') = \exp(-ie  \int_{0}^{1} A_\mu (x' + s(x -x')) (x-x')^{\mu}ds).
%\end{align}
%\note{check definitions}
%\section{The algebra of observables}
%
%\section{The GNS theorem}
%
%\section{Operator valued distributions and two point functions}
%
%\section{Hadamard states}
%
%\section{Point-Split renormalization}
